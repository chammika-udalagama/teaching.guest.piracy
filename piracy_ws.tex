
\PassOptionsToPackage{table}{xcolor}
\documentclass[landscape,DIV=18,a3paper,10pt]{article}
\usepackage{pdfpages}
\usepackage{tikz}
\usepackage{etoolbox}
\usepackage[left=1cm,right=1cm,top=1cm,bottom=1cm]{geometry}
\usepackage{bookman}
\usepackage{multicol}
\usepackage{amsmath}

%:--------------------------------------------------
%:new commands
%:--------------------------------------------------

%:-----------------answers----------------------
\newcommand{\answer}[2][0.125]{
\ifbool{showAll}{%true
\underline{\parbox{#1\linewidth}{\color{myGreen}\centering #2}}}
{%false
\underline{\tikz{\useasboundingbox (0,0)--+(#1\linewidth,0);}}}
}

\newcommand\abox[1]{
\ifbool{showAll}{%true
{\small\color{myGreen}{#1}}
}{%false
\smallskip
\tikz{
%\useasboundingbox (0,0) rectangle +(0.9\linewidth,0.1\textheight);
\draw (0.45\linewidth,0.05\textheight)node[text width=0.9\linewidth,fill=none,draw=none]{\color{white}#1};
}\smallskip

}}

\newcommand{\ru}[1]{\rule{#1}{1pt}}

\newcommand{\myCBox}[1]{\fcolorbox{gray!25}{white}{#1}}

\def\br{\\[10pt]}
\def\deg{^{\circ}}
\newbool{showAll}
\booltrue{showAll}
\boolfalse{showAll}

\input{mySharedTeX}

\pagestyle{empty}
\setlength{\parindent}{0pt}


%:################################################################
\begin{document}
%:################################################################

%\includepdf{./done/topic_01_map.pdf}
\newgeometry{left=0cm,right=1cm,top=1cm,bottom=0.75cm}

\begin{center}

\medskip

\huge \textit{\color{myRed}\bfseries Basic Ocean Navigation} \\\textit{\large \color{myGreen}For Aspiring Pirates, Privateers and other such Adventurers}

%\rule{0.5\linewidth}{1pt}
\rule{1.1\linewidth}{1pt}\\


\medskip
%\rule{0.75\linewidth}{1pt}
\end{center}

%\small
%\begin{minipage}{0.05\linewidth}
%\
%\end{minipage}
%\begin{minipage}{0.6\linewidth}
%%	\section*{A few questions:}
%	\begin{multicols}{2}[\section*{A few questions:}]
%	\begin{itemize}
%		\item The Earth looks flat. How can you prove that it is not?
%		\item The Earth seems to be huge. How will you estimate its size?
%		\item How will you draw map of the Earth (or easier of NUS)?
%		\item There is so much space around us. 
%			\begin{itemize}
%		\item how do we know where we are?
%		\item how do we get from one point to another?
%	\end{itemize}
%
%	\end{itemize}
%	\end{multicols}	
%\end{minipage}
%\begin{minipage}{0.4\linewidth}
%
%\end{minipage}

\vspace{1cm}
\rule{1.1\linewidth}{1pt}\\

\includegraphics[width=\linewidth,clip=true,trim=0cm 0.065cm 0cm 0cm]{./figures/how_World_Map_pacific_centred_hiRes_modified.png}

\restoregeometry

%\begin{multicols}{2}
\twocolumn
%%:--------------------------------------------------
%\section*{Shadows}
%%:--------------------------------------------------
%
%\begin{enumerate}\resume
%%\item What were the lengths of your shadow?\br
%%\begin{tabular}{ccccccccc}
%%10 am & \ru{2cm} & 12-noon & \ru{2cm} & 2 pm  & \ru{2cm}
%%\end{tabular}
%
%%\item The length of your shadow at noon is not always zero. What does this tell us?\br \answer[0.85]{The Sun is \textbf{not} directly above our head.}
%
%\item The digram below shows sunlight impinging on the Earth. 
%\begin{enumerate}
%\item Indicate the length of the shadows of the poles that are positioned at different locations on the Earth.
%\item Indicate the angles between the sunlight and the pole as $a$, $b$ and $c$. 
%\item Indicate and express the angles $\angle AOB$ and $\angle AOC$ using either $a$, $b$ or $c$. 
%\smallskip
%
%\begin{center}
%\begin{tabular}{cccc}
%$\angle COB:$ & \answer{b}& $\angle COA:$ & \answer{c}
%\end{tabular}
%\end{center}
%
%\end{enumerate}
%\bigskip
%
%\begin{center}
%\begin{tikzpicture}[scale=0.75]
%\def\cen{(0.65*\linewidth,\r)}
%\def\r{5}
%\def\h{1.5}
%
%\newcommand\obj[2]{
%\draw[ultra thick,myBrown] \cen  ++({#1}:\r)node[black,right]{\scriptsize #2}--+({#1}:\h) ;
%\draw[dashed,thin,opacity=0.25] \cen -- +(#1:\r);
%}
%
%\useasboundingbox (0,0) rectangle +(\linewidth,2*\r);
%\draw[opacity=0.125,dashed] (0,0) grid (0.975*\linewidth,2*\r);
%\draw[draw=black,fill=blue!5] \cen node[right]{O}  circle (\r);
%\obj{180}{C}
%\obj{160}{B}
%\obj{140}{A}
%\foreach \y in {0,...,10}{
%\draw[thick,->,yellow!80!black] (0,\y) -- + (0.2*\linewidth,0);
%};
%\draw (0.25*\r,1.1*\r) node[yellow!80!black,fill=white]{\textsc{\scriptsize Sunlight}};
%\end{tikzpicture}
%\end{center}
%
%\end{enumerate}\save
%
%%:--------------------------------------------------
%\section*{The size of the Earth}
%%:--------------------------------------------------
%\begin{multicols}{2}
%%\begin{minipage}{0.5\linewidth}
%%\[
%%\rowcolors{2}{gray!5}{white}
%%\begin{array}{rcc}
%%\text{Angle}  & & \text{Length of Arc}  \\
%%\toprule
%%360\deg & \rightarrow & 2\pi R \br
%%180\deg & \rightarrow & \pi R \br
%%90\deg & \rightarrow &\answer{$\dfrac{1}{2}\pi R$}  \br
%%45\deg & \rightarrow & \dfrac{1}{4}\pi R \br
%%30\deg & \rightarrow & \answer{$\dfrac{1}{6}\pi R$}  \br
%%10\deg & \rightarrow & \dfrac{1}{18}\pi R \br
%%1\deg & \rightarrow & \answer{$\dfrac{1}{180}\pi R$}  \br
%%x\deg & \rightarrow & \answer{$\dfrac{x}{180} \pi R$}  \br
%%\bottomrule
%%\end{array}
%%\]
%Eratosthenes' estimate of the Earth's circumference was based on the idea of the circumference of a circle and the shadow cast by the Sun at different locations of the Earth. So, lets first get to know the circle better\ldots
%
%\begin{enumerate}
%
%\item If the radius of a circle is $R$ how long is its circumference?\br \answer[0.75]{$2\pi R$}
%
%%\item A `whole' circumference corresponds to 360$\deg$. What arc lengths corresponds to other angles?
%
%\item Using the triangles $COB$ and $COA$ and taking the radius of the Earth as $R_E$, write down the distance $CB$ and $CA$.
%\begin{center}
%\begin{tabular}{cccc}
%$CB:$ & \answer[0.25]{$\dfrac{b}{360}\times 2\pi R_E$}& $CA:$ & \answer[0.25]{$\dfrac{a}{360}\times 2\pi R_E$}
%\end{tabular}
%\end{center}
%
%\begin{tikzpicture}[scale=0.5]
%\def\cen{(0.5*\linewidth,0.15*\textheight)}
%\def\r{3}
%\useasboundingbox (0,0) rectangle +(0.5*\linewidth,0.3*\textheight);
%\draw[draw=black] \cen circle (\r);
%\draw[draw=none,fill=gray!10] \cen -- ++ (30:\r) arc (30:75:\r) -- cycle;
%\draw[dashed, thick] \cen -- ++ (30:\r) arc (30:75:\r) -- cycle;
%\draw[draw=myBrown,ultra thick] \cen  ++ (30:\r) arc (30:75:\r) ;
%\draw[thick,<->,myGreen] \cen  ++ (30:0.25*\r) arc (30:75:0.25*\r);
%\draw \cen +(50:0.35*\r) node{$\theta$};
%\draw \cen +(50:1.55*\r) node{$\dfrac{\theta}{360}\times 2\pi R$};
%\draw[thick,->,gray] \cen +(-155:1.45*\r) node[below]{\it \scriptsize \textsc{Circumference}} arc(180:100:1);
%\draw[thick,->,gray] \cen +(15:1.3*\r) node[below]{\it \scriptsize \textsc{Arc}} arc(0:110:1);
%\end{tikzpicture}
%
%\item Lets assume $A$ to be Alexandria and $C$ to be Syene. 
%\newline Eratosthenes figured out that the angle formed by the shadow at Alexandria (i.e. $a$) is $7^\circ$ and that the distance ($CA$) between the cities is $785$~km. 
%\newline Use this information to determine the radius $R_E$ of the Earth.
%
%\abox{
%\small
%\begin{align*}
%CA &= \dfrac{a}{360}\times 2\pi R_E\br
% \Rightarrow R_E &= \dfrac{360}{a}\times \dfrac{1}{2 \pi} \times CA = \dfrac{360}{7}\times \dfrac{1}{2 \pi} \times 785\text{ km}   
%\approx 6204\text{km}
%\end{align*}
%}
%
%\item Google for the current value of the Earth's radius. $R_E$ \answer{6371}{3cm}~km.
%
%(Notice how close Eratosthenes got to the real value over, 2000 years ago!!!)
%\save
%\end{enumerate}
%\end{multicols}
%
%\newpage
{
%:--------------------------------------------------
\section*{A few things to ponder upon\ldots}
\color{blue!50!black}
	\begin{itemize}
		\item The Earth looks flat. How can you prove that it is not?
		\item The Earth seems to be huge. How will you estimate its size?
		\item How will you draw map of the Earth (or easier, of NUS)?
		\item There is so much space around us;
		\begin{itemize}
			\item how do we know where we are?
			\item how do we get from one point to another?
		\end{itemize}
		\item Is a compass enough for you to find your way?
		
	\end{itemize}
}

%:--------------------------------------------------
\section*{Where am I?}
%:--------------------------------------------------

\subsection*{In the Northern Hemisphere}
%\vspace{1cm}
\begin{minipage}{0.45\linewidth}
\begin{enumerate}\addtolength{\itemsep}{5pt}
\item What is the significance of the North star (Polaris)?\br\answer[0.9]{The Earth's axis of rotation passes through the pole star.}
\item Use the diagram to determine our latitude using the North Star.
\end{enumerate}\save

\end{minipage}

\begin{tikzpicture}[
equator/.style={dashed, myBrown!50!black},
axis/.style={dashed, myBlue},
%yshift=0cm,
scale=0.9
]
\def\cen{(0.75*\linewidth,0.7*\r)}
\def\r{4.25}
\useasboundingbox (0,-1) rectangle +(\linewidth,1.5*\r);
%\draw (0,0) rectangle +(\linewidth,2*\r);
%earth
\draw[fill=blue!5] \cen circle(\r);
\draw \cen node[below=0.25*\r cm,left=0.5*\r cm]{\scriptsize Earth};
%
%equator
\draw[equator] \cen -- +(-23.5:1.25*\r)node[right]{\scriptsize Equator};
\draw[equator] \cen -- +(-23.5:-1.25*\r);
%
%axis
\draw[axis] \cen -- +(66.5:2.25*\r)node[text width=3cm,align=center,bleudefrance]{\Large $\star$}node[above, black]{\scriptsize North star, Polaris};
\draw[axis] \cen -- +(66.5:1.5*\r) node[fill=white,rotate=-23.5]{\large $\approx$};
\draw[axis] \cen -- +(66.5:-1.25*\r)node[right]{\scriptsize Axis};;
\end{tikzpicture}

\subsection*{Time \& Longitude}

\begin{enumerate}\resume\addtolength{\itemsep}{5pt}
\item Indicate longitude of the following locations:
\begin{center}
\begin{tabular}{cccccc}
 London& \answer{0.3\degree~E}& San Francisco & \answer{122.4\degree~W}&  Singapore & \answer{103.9\degree~E}\\
\end{tabular}
\end{center}

\item The length of your shadow at noon is not always zero. What does this tell us?\br \answer[0.85]{The Sun is \textbf{not} directly above our head.}
%
\item Indicate the order in which the `new' Sun is observed at the following locations:
\begin{center}
\begin{tabular}{cccccc}
 London& \answer{2}& San Francisco & \answer{3}&  Singapore & \answer{1}\\
\end{tabular}
\end{center}


%
\myCBox{\begin{minipage}{0.45\linewidth}
\item Approximately how long will it take `noon' to reach \underline{London} after Singapore?\\
\abox{
\scriptsize
\[
\begin{array}{R{0.55\linewidth}cl}
Time taken by the Sun to `move' from one meridian to the next (i.e. to move by 1\degree) & = & 4~\text{min} \br
Number of meridians between Singapore \& London &= &103.9^{\circ}~\text{E}  - 0.3^{\circ}~\text{E} \\
&=&103.6^{\circ}\\
Time necessary for the Sun to move through $103.6^{\circ}$ & = & 4~\frac{\text{min} }{^{\circ}} \times 103.6^{\circ}\\
&=&414.4~\text{min}\\
&\approx&6.9~\text{h}\\
\end{array}
\]
}
\hfill\scriptsize(answer: 6.9~h)
\end{minipage}
}
%
\hspace{0.05\linewidth}
\myCBox{\begin{minipage}{0.45\linewidth}
\item Approximately how long will it take `noon' to reach \underline{San Francisco} after Singapore?\\
\abox{
\scriptsize
\[
\begin{array}{R{0.55\linewidth}cl}
Time taken by the Sun to `move' from one meridian to the next (i.e. to move by 1\degree) & = & 4~\text{min} \br
Number of meridians between Singapore \& San Francisco &= &103.9^{\circ}~\text{E}  + 122.4^{\circ}~\text{W} \\
&=&226.3^{\circ}\\
Time necessary for the Sun to move through $103.6^{\circ}$ & = & 4~\frac{\text{min} }{^{\circ}} \times 103.6^{\circ}\\
&=&905.2~\text{min}\\
&\approx&15.1~\text{h}\\
\end{array}
\]
}
\hfill\scriptsize(answer: 15.1~h)
\end{minipage}}
\newpage
\item Mr Bond; Mr James Bond, is in trouble. He just woke up to find himself floating in the middle of the ocean. Lucky for him the fancy tuxedo that Q had given him is keeping him warm and afloat. All he remembers is going out to drinks with his `friends' in London, and he is now wondering where he is. He notices that the Sun is not casting a shadow and therefore figures out it should be time for lunch. However, when he looks at his watch, it shows 2 pm! What is Mr Bond's present longitude?

\myCBox{\abox{
Mr. Bond's watch shows the time in London. If it is 2~pm in London, that means the Sun has moved two-hours worth of meridians towards the west from the location of London. Since, every how is worth 15\degree, Mr. Bond is at 30\degree E; i.e. in the middle of the Atlantic!
\bigskip
}}
\end{enumerate}\save

%\newpage
%----------------------------------------------------------------------
\section*{Atmospheric \& Ocean Circulation}
%----------------------------------------------------------------------
\begin{enumerate}\resume\addtolength{\itemsep}{5pt}
\item Air is made-up of mostly nitrogen (79\%, N$_2$, atomic mass 14.01) and oxygen (21\%, O$_2$, atomic mass 16.00). Water is made up of one oxygen atom and two hydrogen, (H, atomic mass 1.008) atoms.\\ Which is `heavier': dry air (i.e. without any water) or air with water?

\myCBox{\abox{
\begin{tikzpicture}
\def\w{\linewidth}
\def\h{0.05\textheight}
\useasboundingbox (0,0) rectangle (\w,\h);
\draw (0,0) rectangle (\w,\h);
\end{tikzpicture}
}}

\item More \answer[0.2]{dense} material float above \answer[0.2]{less} dense material.

\item An example: \answer[0.2]{Oil \& Water, Ice \& Water}

\item Convection is the process where \answer[0.2]{warmer} material move \answer[0.2]{upwards}. 

\item When a gas is allowed to expand, it \answer[0.25]{cools down}.

\item Atmospheric pressure \answer[0.25]{decreases} as you move upwards.

\item Use the digram below to show the workings of the atmospheric convective cells.
\medskip

\includegraphics[width=\linewidth,clip=true,trim=0cm 0cm 0cm 0cm]{./figures/t11_08p238_f16a_mod.png}
%\end{enumerate}\save

%----------------------------------------------------------------------
%\subsection*{Coriolis}
%----------------------------------------------------------------------
%\begin{enumerate}\resume
\item The Coriolis effect causes objects moving in the Northern hemisphere to be deflected to the \answer{right} and those moving in the Southern hemisphere to be deflected to the \answer{left}.

\item The \answer{strength} of the Coriolis effect \answer{increases} with latitude.
%
\item There is \answer{no} Coriolis deflection at the \answer{equator}.
%
%%\item Indicate the direction and the amount of deflection due to the Coriolis effect. Assume you are in the Northern hemisphere. % and note that the green region is at a higher latitude than the yellow region. 
%
%\item Draw in which direction will each of the lines curve?
%\begin{center}
%\begin{tikzpicture}
%%\useasboundingbox (-\linewidth/2,-\textheight/2) rectangle (\linewidth/2,\textheight/2);
%\draw[very thick,<-] (-1,0) -- ++(-2,0);
%\draw[very thick,<-] (1,0) -- ++(2,0);
%\draw[very thick,<-] (0,1) -- ++(0,2);
%\draw[very thick,<-] (0,-1) -- ++(0,-2);
%\draw (0,-3.5)node{Clockwise};
%
%\draw[very thick,<-] (-10,0) -- ++(-2,0);
%\draw[very thick,<-] (-8,0) -- ++(2,0);
%\draw[very thick,<-] (-9,1) -- ++(0,2);
%\draw[very thick,<-] (-9,-1) -- ++(0,-2);
%\draw (-9,-3.5)node{Counter Clockwise};
%\end{tikzpicture}
%\end{center}
\end{enumerate}\save

\end{document}
