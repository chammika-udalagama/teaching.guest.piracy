/mnt/theChest/my-bin/tex/my-beamer.tex

\renewcommand\howTitle[3]{
	\title[GEH1013]{\large \bfseries \color{black}GEH1013\\\medskip}
	\subtitle{{{\fontsize{30}{150}\selectfont #1}}\\\smallskip \textcolor{black}{\itshape \small #2}}
	\author[Chammika Udalagama]{\scriptsize \href{mailto:chammika@nus.edu.sg}{\textcolor{green!50!black}{Chammika Udalagama}}
	}
	\date{2021}

	\addtocounter{framenumber}{-3}
}

% ------------------------------------------------------------------------



% ------------------------------------------------------------------------

\howTitle{Basic Ocean Navigation}{For Aspiring Pirates, Privateers and other Adventures\ldots}{Guest Lecture}

\begin{document}
%\begin{frame}
%\begin{enumerate}
%\item Bugs movie
%%\includemovie[poster,text={\img}]{0.9\linewidth}{0.55\textheight}{./videos/Bugs_Bunny_Hare_We_Go_1951.mp4}\vspace{1pt}}{
%\item Eratos video
%%http://www.youtube.com/watch?v=8On7yCU1EjQ
%
%\end{enumerate}
%\end{frame}

{
	\setbeamertemplate{footline}{}
	\setbeamertemplate{headline}{}
	\begin{frame}
	\begin{tikzpicture}
	\useasboundingbox (0,0) rectangle (\textwidth,\textheight);
	\draw[fill=black] (-.125\textwidth,-.2\textheight) rectangle (1.125\textwidth,1.15\textheight);
	\draw (.595\textwidth,.475\textheight) node[text width = \linewidth]{\includegraphics[height=\textheight]{how_ecl-lun-2008-08-16-umbral-shadow.jpg}};
	\end{tikzpicture}
\end{frame}
}

%------------------------------------------------------------%
%                                                            %
%                        Quote Slide                         %
%                                                            %
%------------------------------------------------------------%
{
\setbeamertemplate{footline}{\logo}
\setbeamertemplate{headline}{}
\begin{frame}
\medskip
\begin{quotation}
	``On my honour as a member of the Democratic Order of Pirates International (D.O.P.I), I promise to be greedy, tricky, mean and icky\ldots''
\end{quotation}
\begin{flushright}
	`The Pirates Pledge'\\\small from the cartoon ``\textit{The Adventures of Dr.Doolittle}'' (1970) \\[5pt]
	\tiny (Watch \href{https://youtu.be/SQtMMYGTNz8}{Dr. Doolitle} on YouTube)
\end{flushright}



\end{frame}
}

%------------------------------------------------------------%
%                                                            %
%                        Title slide                         %
%                                                            %
%------------------------------------------------------------%
{
\setbeamertemplate{footline}{\logo}
\begin{frame}
\maketitle

\begin{center}
	\small Please visit:\\
	\href{https://chammika-udalagama.github.io/teaching.guest.piracy/}{chammika-udalagama.github.io/teaching.guest.piracy}
\end{center}
\end{frame}
}
%------------------------------------------------------------%
%                                                            %
%                           Slide                            %
%                                                            %
%------------------------------------------------------------%
\begin{frame}
\frametitle{Lesson Objectives}
\begin{columns}
	\column{.575\linewidth}
	In this lecture we will visit the following topics and questions:
	\begin{itemize}
		\item Can we convince ourselves that the Earth is spherical?
		\item Are there ways to figure out where we are on the planet?
		\item How does the GPS system work?
		\item Appreciate and marvel at the significant challenges faced by the pioneers who took to the sea.
		\item How do the system of parallels, meridians, degrees, minutes and seconds work.
		\item Appreciate that living on a sphere brings about subtle challenges in measuring distances, cartography and timekeeping.
		\item Understand how just living on a ball affects climate.
	\end{itemize}
	\column{.425\linewidth}
	\fig{\includegraphics[width=.875\linewidth, clip=true,trim=0cm 0cm 0cm 0cm]{Analemma_fishburn.jpg}}{The path of the sun in the heavens.}{\href{https://en.wikipedia.org/wiki/Analemma}{Wikipedia}}
\end{columns}
\end{frame}

%------------------------------------------------------------%
%                                                            %
%                           Slide                            %
%                                                            %
%------------------------------------------------------------%
\begin{frame}
\frametitle{The World, She is Flat\ldots}
\def\h{0.575\textheight}

\begin{columns}[b]
\column{0.45\linewidth}
\fig{\includegraphics[height=\h]{t03_Buggs_flat.jpg}}{Bugs Bunny proves the world is round (Watch the video \href{http://youtu.be/U1IuLlU1gjc}{here})}{}
\column{0.55\linewidth}
\fig{\includegraphics[height=\h,clip=true,trim=0cm 0cm 0cm 0cm]{how_1680x1050_discworld-terry-pratchett-wallpaper-e1367902680361.jpg}}{Terry Practhett's `Discworld'.}{ \href{http://zayngotnews.files.wordpress.com/2013/05/1680x1050_discworld-terry-pratchett-wallpaper-e1367902680361.jpg}{www.gopixpic.com}}
\end{columns}

\bigskip

\question{How will you convince a \href{https://en.wikipedia.org/wiki/Modern_flat_Earth_societies}{FlatEarther} that the Earth is spherical and not flat like `Discworld'?\\ (Perhaps not the way Bugs does\ldots \smiley)}
\end{frame}

%------------------------------------------------------------%
%                                                            %
%                           Slide                            %
%                                                            %
%------------------------------------------------------------%
\begin{frame}
\frametitle{Did You Know that the Earth is a Ball?}
\def\h{0.55\textheight}

\begin{columns}[b]
	\column{.5\linewidth}
	\fig{\includegraphics[height=\h]{t03_shipOver.jpg}}{A ship `disappearing' over the horizon.}{}
	\column{.5\linewidth}
	\fig{\includegraphics[height=\h,clip=true,trim=0cm 0cm 0cm 0cm]{how_ecl-lun-2008-08-16-umbral-shadow.jpg}}{Photos of Earth's Shadow.}{\href{http://www.perseus.gr/}{Anthony Ayiomamitis}}
\end{columns}

\question{Can you think of a way to estimate the size of the Earth?}
\end{frame}

%------------------------------------------------------------%
%                                                            %
%                           Slide                            %
%                                                            %
%------------------------------------------------------------%
\begin{frame}
\frametitle{Eratosthenes}

\fig{\includegraphics[width=0.3\linewidth]{t03_Portrait_of_Eratosthenes.png}}{Portrait of Eratosthenes}{\href{http://en.wikipedia.org/wiki/Eratosthenes}{Wikipedia}}

\begin{itemize}
\item \href{http://en.wikipedia.org/wiki/Eratosthenes}{Eratosthenes of Cyrene} (276 BC - 194 BC) was the second librarian of the great\\ Library of Alexandria.
\item Eratosthenes figured out that the Earth was spherical and estimated the Earth's circumference to within 8\% of its real value! This was over 2000 years ago!!!
\item Watch	\href{http://www.youtube.com/watch?v=8On7yCU1EjQ}{this video} to figure out how he did this.
\end{itemize}
\end{frame}

%------------------------------------------------------------%
%                                                            %
%                           Slide                            %
%                                                            %
%------------------------------------------------------------%
\begin{frame}
\frametitle{The Size of the Earth}
\begin{itemize}
\item Eratosthenes used his keen powers of observation and a lot of ingenuity to calculate the circumference of the earth.

\item He just used a deep well, the shadow of a tall pillar, and some simple geometry.
\newline Let's see how his method works\ldots
\end{itemize}
\medskip

\fig{\includegraphics[width=0.8\linewidth]{t03_02p036_f03.jpg}}{The idea behind Eratosthenes' method to determine the circumference of the Earth.}{\garrison}

\bigskip

\question{If Eratosthenes had been in Singapore (\smiley), how well do you think would have his method worked?}
\end{frame}

%------------------------------------------------------------%
%                                                            %
%                           Slide                            %
%                                                            %
%------------------------------------------------------------%
\begin{frame}
\frametitle{Finding Our Way Round}

\begin{columns}
\column{.5\linewidth}
\begin{itemize}
\item The Earth, being a ball, has no boundaries, so specifying locations is not obvious. So we use a `curved' grid to help us pin-point places.
\item This `grid' is made up of \textbf{latitudes} (parallels) and \textbf{longitudes} (meridians).
\item  In this scheme the Earth is split into 180 {latitudes} (\textbf{parallels}) and 360 {longitudes} (\textbf{meridians}).
\end{itemize}

\column{.5\linewidth}
\def\h{.475\textheight}
\fig{\includegraphics[height=\h,clip=true,trim=0cm 0cm 0cm 0cm]{t03_App3p553_f01.png}
\includegraphics[height=\h,clip=true,trim=0cm 0cm 0cm 0cm]{t03_App3p554_f03.png}
}{There are 180 latitudes\,(also called parallels) and 360 longitudes\,(also called meridians).}{\garrison}
\end{columns}
\end{frame}

%------------------------------------------------------------%
%                                                            %
%                           Slide                            %
%                                                            %
%------------------------------------------------------------%
\begin{frame}
\frametitle{Where is Zero?}

\begin{itemize}
\item An obvious choice for zero latitude (parallel) is the \textbf{equator}.
\item There is no obvious choice of zero for longitudes (meridians). It is agreed that the \textbf{prime meridian} (i.e. zero longitude) is the one through Greenwich, England\footnote{This was not always the case (see this  \href{https://en.wikipedia.org/wiki/Prime_meridian}{Wikipedia page}).}.
\end{itemize}

\fig{\centering\includegraphics[width=.675\linewidth, clip=true,trim=0cm 0cm 0cm 0cm]{t03_02p037_u01.jpg}}{How latitudes and longitudes are specified.}{\garrison}
\end{frame}

%------------------------------------------------------------%
%                                                            %
%                           Slide                            %
%                                                            %
%------------------------------------------------------------%
\begin{frame}
\frametitle{Parallels \& Meridians}
\begin{columns}
	\column{0.6\linewidth}
	\begin{itemize}
		\item Latitudes and longitudes are given in degrees.
		\item We also use \textbf{N}~or~\textbf{S} for latitudes and \textbf{E}~or~\textbf{W} for longitudes.
		\item E.g. Singapore is approximately at:
		\begin{center}
			1\degree N 103\degree E
		\end{center}
		\item Further refinement is obtained by using minutes ($'$) and seconds ($''$).
		\item $1^{\circ} = 60'$ and $1' = 60''$
		\item E.g. Singapore is `exactly' at:
		\begin{center}
			1\degree 17`\,N~103\degree 50`\,E
		\end{center}
	\end{itemize}

	\bigskip

	\question{Where is this `exact' point in Singapore? Google to find out!}
	\column{0.35\linewidth}
	\fig{\includegraphics[clip=true, trim=35cm 7.25cm 0cm 0cm,width=\linewidth]{t03_02p037_u01.jpg}}{Location can be specified using angles.}{\garrison}
\end{columns}
\end{frame}

%------------------------------------------------------------%
%                                                            %
%                           Slide                            %
%                                                            %
%------------------------------------------------------------%
\begin{frame}
\frametitle{This Minute is Not the Same as that Minute\ldots}
\medskip
\fig{\begin{tabular}{ccc}
	\includegraphics[height=0.19\textwidth]{t03_clock.jpg}
	&\includegraphics[height=0.15\textwidth]{t03_Unequal.jpg}
	&\includegraphics[height=0.19\textwidth]{t03_protractor.jpg}
\end{tabular}}{Minutes~($'$) and Seconds~($''$) are measures of angles, not time!}

\begin{itemize}
\item Degrees~(\degree), Minutes~($'$) and Seconds~($''$) are use to measure \textbf{angles}.
\item If we used the more familiar decimal system:
%
{\small\begin{align*}
	1' &= \left(\frac{1}{60}\right)^\circ = 0.017^\circ\\
	1'' &= \left(\frac{1}{60}\right)'=\left(\frac{1}{60}\times\frac{1}{60}\right)^\circ = \left(\frac{1}{3,600}\right)^\circ = 0.00028^\circ\\
	10^\circ 30' &= 10.5^\circ
	\end{align*}}
%
\item Notice that it is easier to use degrees, minutes and seconds that its decimal equivalent.
\end{itemize}
\end{frame}

%------------------------------------------------------------%
%                                                            %
%                           Slide                            %
%                                                            %
%------------------------------------------------------------%
\begin{frame}
\frametitle{Where in the World Are We?!}
\medskip
\begin{columns}[b]
\column{0.5\linewidth}
\begin{itemize}
\item The North Star\,(Polaris), can be used to determine latitude of a location in the \textbf{northern} hemisphere.
\end{itemize}

\fig{\includegraphics[width=0.9\textwidth]{t03_App3p555_f05.jpg}}{The North Star can be used to determine \textcolor{red}{latitude}.}{\garrison}

\column{0.5\linewidth}
\fig{\includegraphics[width=0.925\textwidth]{t03_polarstern.jpg}}{The Earth's axis of rotation points at the North Star.}{\garrison}
\end{columns}
\end{frame}

%------------------------------------------------------------%
%                                                            %
%                           Slide                            %
%                                                            %
%------------------------------------------------------------%
\begin{frame}
\frametitle{Using the North Star for Latitude}
\medskip

\begin{itemize}
\item Its all in the angles\ldots
\end{itemize}

\fig{\begin{tikzpicture}[thick,scale=0.7575]
%\useasboundingbox (-\linewidth/2,0\textheight) rectangle (\linewidth/2,\textheight/2);
\draw (0,0) node{\centering{\includegraphics[width=0.525\textwidth]{t03_App3p555_f05.jpg}}};
\draw[dashed,thick, yellow] (-3.045,-0.1)--++(-90:3)--++(0:2.9)--cycle;
\draw[dashed, myGreen] (-3.045,-0.1)--+(-45.75:-1);
\draw[myRed] (-3.045,-3.1) rectangle +(0.125,0.125);
\draw[myRed,rotate=41.75] (-2.455,1.85)rectangle +(0.125,0.125);
\draw[myRed,<->] (-3.045,0)++(90:0.75)arc(-90:-148:-0.75);
\draw[myRed,<->] (-3.045,0)++(-90:0.75)arc(-90:-137:0.75);
\draw[myRed,<->] (-3.045,0)++(-47:3.6)arc(-50:-2.5:-0.65);
\end{tikzpicture}
}{The angles that help us to determine \textbf{latitude} using Polaris}{\garrison}

\bigskip
\question{What do we do in the southern hemisphere? Is there a `South' star?}
\end{frame}

%------------------------------------------------------------%
%                                                            %
%                           Slide                            %
%                                                            %
%------------------------------------------------------------%
\begin{frame}
\frametitle{Importance of Longitude}
\begin{columns}
	\column{0.6\linewidth}
	\begin{itemize}
		\item Even if you knew north/south and latitude, travelling the ocean without knowing your longitude is dangerous.

		\begin{flushleft}\small
			\begin{quotation}
				``For lack of a practical method of determining longitude, every great captain in the Age of Exploration became lost at sea despite the best available charts and compasses. From Vasco da Gama to Vasco N\'u\~{n}ez de Balboa, from Ferdinand Magellan to Sir Francis Drake--they all got where they were going willy-nilly, by forces attributed to good luck or the grace of God.''
			\end{quotation}
			\hfill From Dave Sobel's `Longitude...'~\cite{Sobel1995}
		\end{flushleft}
		\item A storm or current can easily make you lose your bearing.
		\item The `longitude problem' cost governments lots of lives and money that in 1714 the British government passed an act of parliament that offered a (staggering) prize of \pounds20,000 for an accurate solution.
	\end{itemize}
	\column{0.3\linewidth}
	\medskip
	\fig{\includegraphics[width=0.8\linewidth]{t03_global_latitude.png}\newline		\includegraphics[width=0.9\linewidth]{t03_global_longitude.png}}{The heavenly bodies can help with latitude but not with longitude.}{}
\end{columns}
\end{frame}

%------------------------------------------------------------%
%                                                            %
%                           Slide                            %
%                                                            %
%------------------------------------------------------------%
\begin{frame}
\frametitle{Time \& Longitude}
\def\fn{\tiny Assuming the time is based on the Sun.}
\begin{columns}[t]
	\column{0.5\linewidth}%
	\begin{itemize}
		\item Since there are 360\degree meridians and the Earth rotates through all these meridians in 24~hours:
		{
			\color{black!90}
			\small\begin{align*}
			&\text{\shortstack[l]{Time for the Sun `to go' from\\ one meridian (1\degree) to the next}} \\
			&= \frac{24\text{ h}}{360 ^\circ} \\
			&= \frac{24\times 60 \text{ min}}{360^\circ} \\
			&= 4\text{ min/\degree}
			\end{align*}
		}
	\end{itemize}
	\column{0.5\linewidth}
	\fig{\includegraphics[width=.9\textwidth,clip=true,trim=0cm 0cm 0cm 0cm]{how_sunmap.jpeg}}{The time of day (usually) depends on the position of the Sun. Look \href{http://www.timeanddate.com/worldclock/sunearth.html}{here} to see where the Sun is now!}{ \href{http://www.timeanddate.com/worldclock/sunearth.html}{Day and Night World Map}}
\end{columns}

\begin{itemize}
	\item So, if we know the time at our location, we can figure out\footnote{\fn} the time at another location if we know its longitude.
	\item Also, if we know the times$^\text{1}$ at two locations, then we can figure out the difference in longitude between these two locations.
\end{itemize}
\end{frame}

%------------------------------------------------------------%
%                                                            %
%                           Slide                            %
%                                                            %
%------------------------------------------------------------%
\begin{frame}
\frametitle{Noon is Not Just Only for Lunch\ldots}
\fig{\includegraphics[width=.5\textwidth]{t03_shadow.jpg}}{The position of the Sun when it casts the shortest shadow signifies `noon'.}{}

\begin{itemize}
\item Noon marks the point when the Sun is at its peak, for that day.
\item Noon is when the shadow cast by the Sun is the shortest (not necessarily zero; why?).
\item This unique position of the Sun can be used to standardize what time is.
%\item \color{red} E.g. If you are at the 120\s{$\circ$}~W meridian, noon would have occurred at Greenwich 480~min ago. So, your are in the -8~h time zone.
%\item \color{red} Conversely, if the time at Greenwich is known when noon occurs at your location, say 13:00, then we must be at 1~h west or 15\s{$\circ$}~W of Greenwich.
\end{itemize}
\question{ Can `sunrise' or `sunset' be used for time-keeping?}
\end{frame}

%------------------------------------------------------------%
%                                                            %
%                           Slide                            %
%                                                            %
%------------------------------------------------------------%
\begin{frame}
\frametitle{Longitude in the Days Past}
\begin{columns}
\column{0.5\linewidth}
\begin{itemize}
\item When you are at sea, you can use the Sun and the stars to determine your latitude, but not longitude.
\item Complicated methods involving the Sun, the Moon and the stars were proposed and even used.
\item The most reliable way was to use `noon' at the present location, along with an accurate clock.
\item The problem was designing and manufacturing a clock that can withstand the vicissitudes of ocean travel.
\end{itemize}
\column{0.5\linewidth}
\fig{\includegraphics[width=0.5\textwidth,clip=true,trim=0cm 0cm 0cm 0cm]{how_Jacobstaff.jpg}}{Using a `\href{http://en.wikipedia.org/wiki/Jacob's_staff}{Jacob's Staff}' to determine the position of the Sun. Lots of people ended up blind in one eye with this instrument.}{\href{http://en.wikipedia.org/wiki/Jacob's_staff}{Wikipedia}: "Jacobstaff" by John Seller (1603-1697) - Scan from the original book Practical navigation (1st edition 1669) p. 200}
\end{columns}
\end{frame}

%------------------------------------------------------------%
%                                                            %
%                           Slide                            %
%                                                            %
%------------------------------------------------------------%
\begin{frame}
\frametitle{H-4}
\begin{columns}
\column{0.65\linewidth}
\begin{itemize}
\item \href{http://en.wikipedia.org/wiki/John_Harrison}{John Harrison}'s fantastic chronometer, that won him the \pounds20,000 prize in 1773.
%
\item This is the fourth and the most accurate that Harrison manufactured.

{\small
\begin{flushleft}
\begin{quotation}
``I think I may make bold to say, that there is neither any other Mechanical or Mathematical thing in the World that is more beautiful or curious in texture than this my watch or Timekeeper for the Longitude . . . and I heartily thank Almighty God that I have lived so long, as in some measure to complete it.''
\end{quotation}
\end{flushleft}
\begin{flushright}
-- John Harrison -- \\
From 'Longitude'~\cite{Sobel1995}
\end{flushright}
}
%
\item There is a very rich historical backstory to this topic involving a lot of great names such as Galileo, Euler, Newton, Halley\ldots (again; see `Longitude'~\cite{Sobel1995}).
\end{itemize}
\column{0.4\linewidth}
\fig{\includegraphics[width=\linewidth]{t03_02p050_f18.jpg}}{Harrison's fourth timepiece: H-4.}{\garrison}
\end{columns}
\end{frame}

%------------------------------------------------------------%
%                                                            %
%                           Slide                            %
%                                                            %
%------------------------------------------------------------%
\begin{frame}
\frametitle{`Naive' \& Real Time Differences}
\def\rc{\rowcolor{PineGreen!3}}
\def\st{\\}
\begin{center}
\includegraphics[width=0.6\textwidth]{t03_world_pol495.png}\\

\small
\renewcommand{\arraystretch}{1.5}
\begin{tabular}{l|rr|r||r|r}
\multirow{2}{*}{City} &Longitude   & Longitude & Longitude & \multicolumn{2}{c}{Time Difference (h)}  \st
&  (' '' \degree)  & ({$^\circ$}) & `Distance' & `Naive' & `Real' (UTC) \\
\toprule
Singapore & $103^\circ 55'$~E &$103.9$~E & $0$ & $0$ & $0$ \st
\rc New York & $73^\circ  56'$~W & $73.9$~W & $177.8$ & $-11.9$ & $-13$ \st
San Francisco & $122^\circ  25'$~W & $122.4$~W & $226.3$ & $-15.1$ & $-16$ \st
\rc London & $0^\circ  15'$~E  &$0.3$~E & $103.6$ & $-6.9$ & $-8$ \st
Shanghai & $121^\circ  28'$~E & $121.5$~E & $17.6$ & $1.2$ & $0$ \\
\end{tabular}
\end{center}
\end{frame}

%------------------------------------------------------------%
%                                                            %
%                           Slide                            %
%                                                            %
%------------------------------------------------------------%
\begin{frame}
\frametitle{(UTC) Time Differences in Real Life}
\medskip
\fig{\includegraphics[width=0.75\textwidth]{t03_Standard_time_zones_of_the_world.png}}{The world has been separated into 26 different time zones. Visit \url{http://www.timeanddate.com/time/map/} for an interactive map.}{}
\begin{itemize}
\item UTC (\emph{Coordinated Universal Time}) times zones extends from -12 to +14 from Greenwich.
\end{itemize}
\end{frame}

%------------------------------------------------------------%
%                                                            %
%                           Slide                            %
%                                                            %
%------------------------------------------------------------%
\def\fn{\tiny Recall what happens in `Around the World in 80 days'?}
\begin{frame}
\frametitle{Where is a `Day' born?}
\begin{columns}
\column{.5\linewidth}
\begin{itemize}
\item A `new day' is born when midnight passes the \textbf{International Date Line}.
\item The date line should ideally follow the 180\degree meridian (it does not).
\item If you cross this line right $\rightarrow$ left (according to the image) you essentially travel into `tomorrow'. So you need to add a day to your watch$^\text{1}$.
\item If you cross this line left $\rightarrow$ right (according to the image) you essentially travel into `yesterday'. So you need to subtract a day from your watch.
\item First place to experience a new day: Kiritimati or Chritmas Island ($157^\circ24"$~\textcolor{myRed}{W}).
\end{itemize}
\column{.5\linewidth}
\fig{\includegraphics[width=0.7\textwidth]{t03_indate.jpg}}{The `International Date Line': where a day is born.}{}
\end{columns}
\textcolor{white}{xx\footnote{\fn}}
\end{frame}

%------------------------------------------------------------%
%                                                            %
%                           Slide                            %
%                                                            %
%------------------------------------------------------------%
\begin{frame}
\frametitle{The Date Line is NOT Straight!}
\begin{columns}
\column{0.6\linewidth}
\begin{itemize}\small
\item Kiritimati or Christmas Island is in the \textbf{Western} hemisphere ($157^\circ24"$~\textcolor{myRed}{W})!
\item This results in a large irregularity in the international date line.
\item The UTC zones +13 and +14 are in this region.
\item Notice how there is a difference of a day between Kiritimati and Hawaii, which is almost on the same meridian!
\end{itemize}
\column{0.35\linewidth}
\fig{\includegraphics[height=0.875\textheight,clip=true,trim=0cm 0cm 0cm 0cm]{b03_International_Date_Line.png}}{The not-so-straight date line.}{Image from the \href{http://en.wikipedia.org/wiki/International_Date_Line}{Wikipedia} article on the International Date Line.}
\end{columns}
\end{frame}

%------------------------------------------------------------%
%                                                            %
%                           Slide                            %
%                                                            %
%------------------------------------------------------------%
\begin{frame}
\frametitle{What is the `Real' Shape of the Earth}
\fig{\includegraphics[width=0.55\textwidth,clip=true,trim=0cm 0cm 0cm 0cm]{b03_actually.jpg}}{The shape of the Earth according to xkcd}{\href{https://xkcd.com}{xkcd}}
\end{frame}


%------------------------------------------------------------%
%                                                            %
%                           Slide                            %
%                                                            %
%------------------------------------------------------------%
\begin{frame}{Map worth \$10,000,000!}{}
\bigskip
	\fig{\includegraphics[width=0.85\textwidth]{./figures/t04_02p047_f14.jpg}}{The Waldseem\"uller map (1507). (See \href{http://en.wikipedia.org/wiki/Waldseemuller_map}{Wikipedia} for more information.)}{\garrison}
\end{frame}

%------------------------------------------------------------%
%                                                            %
%                           Slide                            %
%                                                            %
%------------------------------------------------------------%
\begin{frame}
\frametitle{Looks can be Deceiving}

\begin{columns}[c]
	\column{.5\linewidth}
	\begin{itemize}
		\item Which is bigger, Africa or Greenland?
	\end{itemize}

	\bigskip

	\fig{\includegraphics[width=.85\textwidth]{t04_MollweideProjection_800.png}}{Mollweide Projection}{}
	\column{.5\linewidth}
	\fig{\includegraphics[width=0.85\textwidth]{t04_MercatorProjection_1000.png}}{Mercator Projection}{}
\end{columns}
\end{frame}

%------------------------------------------------------------%
%                                                            %
%                           Slide                            %
%                                                            %
%------------------------------------------------------------%
\begin{frame}
\frametitle{Beware of Greeks Bearing Maps}

\begin{columns}
\column{.5\linewidth}
\begin{itemize}
	\item A (3D) sphere \textbf{cannot} be perfectly projected onto a (flat) plane.
	\item Each projection is optimised for a specific use (e.g. \textbf{directions} for navigation or \textbf{area} for surveying).
\end{itemize}

\fig{\centering\includegraphics[width=.9\linewidth, clip=true,trim=0cm 0cm 0cm 0cm]{map-projections.png}}{There many projections possible. Click \href{https://www.jasondavies.com/maps/transition/}{here} to explore.}{}

\column{.5\linewidth}
\fig{\centering\includegraphics[width=.4\linewidth]{t04_map_comedy_tradegy.png}
	\includegraphics[width=.7\linewidth, clip=true,trim=0cm 0cm 0cm 0cm]{t04_face_projections.png}}{Effect of different projections on the \textbf{comedy} and \textbf{tragedy} faces.}{}

\end{columns}
\end{frame}

%------------------------------------------------------------%
%                                                            %
%                           Slide                            %
%                                                            %
%------------------------------------------------------------%
\begin{frame}
\frametitle{What is the `Real' Shape of the Earth}
\begin{columns}
	\column{0.4\linewidth}
	\fig{\includegraphics[width=\linewidth]{t03_earth-bulge.jpg}}{The Earth is a bit bulging at the waist \smiley.}{}
	\column{0.6\linewidth}
	\begin{itemize}
		\item The actual shape of the Earth is complicated due to its features (valleys, mountains, trenched, \ldots).
		\item The Earth is not really a ball but an oblate (`fatter' in the middle) spheroid.

		\begin{center}
			\begin{tabular}{ccc}
				& Equator & Poles\\
				\toprule
				Diameter & 12,756~km & 12,713~km \\
			\end{tabular}
		\end{center}

		\item This difference of $\approx$ 43~km(!) is a consequence of the Earth's rotation.
		\item Nowadays, the absolute position on Earth is expressed using the World Geodetic System (WGS).
		\item WGS uses fixed points (datums) and gravitational measurements to establish it reference.
		\item GPS plays a prominent role in modern geodacy\\ (see \href{http://www.ngs.noaa.gov}{NGS} and \href{http://geodesy.noaa.gov/CORS/}{CORS}).
	\end{itemize}
\end{columns}
\end{frame}

%------------------------------------------------------------%
%                                                            %
%                           Slide                            %
%                                                            %
%------------------------------------------------------------%
\begin{frame}
\frametitle{Global Positioning System (GPS)}
\bigskip
\fig{\includegraphics[width=0.7\textwidth]{t03_02p061_f35.jpg}}{GPS uses a constellation of satellites.}{}
\begin{itemize}
\item \textbf{GPS} or \textbf{Global Positioning System} was originally called \textbf{NAVSTAR}~-~\emph{Navigation System for Timing and Ranging.}
\end{itemize}
\end{frame}

%------------------------------------------------------------%
%                                                            %
%                           Slide                            %
%                                                            %
%------------------------------------------------------------%
\begin{frame}
\frametitle{GPS - xkcd}
\fig{\includegraphics[width=\textwidth]{t03_xkcd_i_dont_want_directions.png}}{\smiley}{\xkcd}
\end{frame}

%------------------------------------------------------------%
%                                                            %
%                           Slide                            %
%                                                            %
%------------------------------------------------------------%
\begin{frame}
\frametitle{Some Facts about GPS}
\begin{columns}
\column{0.5\linewidth}
\begin{itemize}
\item GPS consisted of 24 satellites (now around 30) and was setup by the US in 1973 for defense purposes.
\item GPS is free but requires a GPS receiver and line-of-sight to the satellites.
\item The orbits of the satellites ensure that there is \textbf{always} at least four in line-of-sight from any point on the globe! (check out \href{http://www.sat-net.com/winorbit/}{WinOrbit})
\item GPS can usually work in all types of weather including heavy cloud cover but not underwater or underground.
\end{itemize}
\column{0.5\linewidth}
\fig{\includegraphics[width=\textwidth]{t03_gps-constellation-of-satellites-580op.jpg}}{You can always see at least four satellites in the NAVSTAR system!}{}
\end{columns}
\end{frame}

%------------------------------------------------------------%
%                                                            %
%                           Slide                            %
%                                                            %
%------------------------------------------------------------%
\begin{frame}
\frametitle{How GPS Works I}
\begin{itemize}
\item The GPS consists of three segments:
\begin{enumerate}
\item \textbf{Space segment}: This is the constellation of satellites, each carrying an atomic clock!
\item \textbf{Control segment}: Ground stations around the globe for monitoring/correcting the orbits.
The main control station is in the Schriever Air Force Base, Colorado.
\item \textbf{User segment}: This is your GPS receiver.
\end{enumerate}
\end{itemize}
%
\begin{columns}
\column{0.6\linewidth}
\begin{itemize}
\item The signal sent by the satellite usually consists of its: \centerline{\color{myGreen}\texttt{[name, position $(x,y,z)$, time ($t_s$)]}}
\item The distance between you and the satellite can be determined by comparing the time $t_{sat}$ and the time ($t_{you}$) in the clock in your GPS receiver.
\begin{equation}
\text{distance} = (t_{you} -t_{sat})\times c
\end{equation}
\begin{center}
	  $c=3.0\times10^8$~m\,s$^{-1}$ is the speed of the (electromagnetic) signal.
\end{center}
\item Because $c$ is so large, even a small inaccuracy in time will result in a big error in distance.
\end{itemize}
\column{0.325\linewidth}
\fig{\includegraphics[width=.9\textwidth]{t03_stopwatch.jpeg}}{Accurate timing is crucial for GPS.}{}
\end{columns}
\begin{itemize}
\item Actually, our watches needs to be as good as an \textbf{atomic clock}(!) which is a problem. \\Luckily there is an ingenious solution!
\end{itemize}

\end{frame}

\def\h{.9\textheight}
%------------------------------------------------------------%
%                                                            %
%                           Slide                            %
%                                                            %
%------------------------------------------------------------%
\begin{frame}
\frametitle{Ingenuity of GPS I}
\fig{\includegraphics[height=\h]{t03_gps_posb_1.png}}{If you have a `good' watch all you need is two satellites.}{}
\end{frame}

%------------------------------------------------------------%
%                                                            %
%                           Slide                            %
%                                                            %
%------------------------------------------------------------%
\begin{frame}
\frametitle{Ingenuity of GPS II}
\fig{\includegraphics[height=\h]{t03_gps_posb_2.png}}{If your watch is `off' you will be at A but think you are at B.}{}
\end{frame}

%------------------------------------------------------------%
%                                                            %
%                           Slide                            %
%                                                            %
%------------------------------------------------------------%
\begin{frame}
\frametitle{Ingenuity of GPS III}
\fig{\includegraphics[height=\h]{t03_gps_posb_3.png}}{With three satellites, even if your watch is `off' you can correct your time until the B's coincide with A!}{}
\end{frame}

%------------------------------------------------------------%
%                                                            %
%                           Slide                            %
%                                                            %
%------------------------------------------------------------%
\begin{frame}
\frametitle{Living on a Ball\ldots is Different}
\def\h{0.35\textheight}
%
\fig{\includegraphics[height=\h]{t03_triangle_angle1.jpg}\hspace{1pt}
	\includegraphics[height=\h]{t03_spherical_triangle.jpg}\hspace{1pt}
	\includegraphics[height=\h]{t03_SmallGreatCircles_700.png}
}{The usual rules of geometry does not seem to work on spheres!}{}
%
\begin{itemize}
	\item Triangles are different, `parallel' is different  \& the shortest distance is \textbf{not} straight.
	\item The shortest distance between two points is along a `\textbf{great circle}'
	\item A `great circle' shares its centre with the centre of the sphere.
	\item This is partly why planes seem to fly `funny' (see \href{http://dynref.engr.illinois.edu/aos.html}{here} for a great simulation).
\end{itemize}

\bigskip

\question{If all this is true, why don't we notice these weird things in everyday life?}
\end{frame}

%%------------------------------------------------------------%
%%                                                            %
%%                           Slide                            %
%%                                                            %
%%------------------------------------------------------------%
%\begin{frame}
%\frametitle{Is a Great Circle Always Possible?}
%\framesubtitle{I.e. can you always have a great circle joining ANY two points on the sphere?}
%\begin{columns}
%	\column{0.5\linewidth}
%		\fig{\includegraphics[width=0.45\textwidth,clip=true,trim=0cm 0cm 0cm 0cm]{b03_globe_two_pts.png}}{You can always draw a plane between any two points on the surface and the centre of the sphere.}{ \href{http://thejuniverse.org/PUBLIC/LinearAlgebra/LOLA/planes/find.html}{LOLA website}.}
%	\column{0.5\linewidth}
%		\fig{\includegraphics[width=0.5\textwidth,clip=true,trim=0cm 0cm 0cm 0cm]{b03_planeEx1.png}}{You can draw only one plane between three points (this is why a tripods work!)}{}
%\end{columns}
%\bigskip
%
%	\begin{itemize}
%	\item You can \textbf{always} draw a plane through \textbf{any} three points (this is why we have \textbf{tri}pods).
%	\item A circle is a collection of  points on a \textbf{plane} that are the same distance from another point; the centre.
%	\item Two points on a sphere (e.g. A and B) is the same distance from the centre. Hence they lie on a circle.
%	\item But wait! The centre of the circle is the centre of the sphere! So, the circle is `great'!
%\end{itemize}
%\end{frame}

%------------------------------------------------------------%
%                                                            %
%                           Slide                            %
%                                                            %
%------------------------------------------------------------%
\begin{frame}
\frametitle{Seasons in the Sun}
\begin{columns}
\column{.5\linewidth}
\begin{itemize}\small
	\item The position of the Sun in the sky changes throughout the year.

	\fig{\includegraphics[width=.35\linewidth, clip=true,trim=0cm 0cm 0cm 0cm]{Analemma_fishburn.jpg}}{The path of the sun in the heavens.}{\href{https://en.wikipedia.org/wiki/Analemma}{Wikipedia}}

	\item The Sun is directly over the equator \textbf{twice} a year. These times are called \textbf{equinoxes}.
	\begin{center}\scriptsize
		\begin{tabular}{ccc}
			& Spring           & Fall             \\
			\toprule
			Equinox & $\approx$ 20 Mar & $\approx$ 20 Sep
		\end{tabular}
	\end{center}

	\item The Sun 'turns around' once it is over the \textbf{Tropic of Cancer}\,(23.5\degree N) or \textbf{Tropic of Capricorn}\,(23.5\degree S).\\ These times are called \textbf{solstices}.
	\begin{center}\scriptsize
		\begin{tabular}{ccc}
			& Summer           & Winter             \\
			\toprule
			Solstices & $\approx$ 20 Jun & $\approx$ 20 Dec
		\end{tabular}
	\end{center}
\end{itemize}
\column{.6\linewidth}
\begin{tikzpicture}
\useasboundingbox (0,0) rectangle (\linewidth,.8\textheight);
\draw (3.5cm,3.5cm) node{\includegraphics[width=9.5cm]{t11_08p233_f08.png}};
\end{tikzpicture}
\fig{}{Seasons are caused due to the Earth's $23\dfrac{1}{2}$\degree tilt.}{\garrison}
\end{columns}
\end{frame}

%------------------------------------------------------------%
%                                                            %
%                           Slide                            %
%                                                            %
%------------------------------------------------------------%
\def\fn{\footnote{\tiny Look \href{http://www.wwnorton.com/college/geo/egeo2/content/animations/18_2.htm}{here} for a cool animation of Milankovitch cycles.}}
\begin{frame}
\frametitle{Angles Are Important}
\fig{\begin{columns}
\column{0.45\linewidth}
\includegraphics[width=\textwidth,clip=true,trim=0cm 0cm 14cm 30cm]{t11_08p231_f06.jpg}
\column{0.55\linewidth}
\includegraphics[width=\textwidth,clip=true,trim=0cm 32cm 1cm 0cm]{t11_08p231_f06.jpg}
\end{columns}\textcolor{white}{.}
}{The angle of incidence of sunlight determines how much light is available for absorption.}{\garrison}

\begin{itemize}
\item There are many other factors (e.g. \href{https://en.wikipedia.org/wiki/Milankovitch_cycles}{Milankovitch cycles}\fn) that affect the amount of light that we receive from the Sun.
\end{itemize}
\end{frame}
%
%%------------------------------------------------------------%
%%                                                            %
%%                           Slide                            %
%%                                                            %
%%------------------------------------------------------------%
%\begin{frame}
%\frametitle{Differential Heating of Earth}
%\fig{
%\begin{tabular}{cc}
%\includegraphics[width=0.55\textwidth,clip=true, trim=0cm 0cm 0cm 5cm]{t11_08p232_f07a.jpg} &
%\includegraphics[width=0.425\textwidth,clip=true, trim=0cm 18cm 0cm 0cm]{t11_08p232_f07b.png}
%\end{tabular}
%}{Averaged over a year; the Sun provides different amounts of heat to the different latitudes of the globe.}{\garrison}
%\begin{itemize}
%\item This differential heating leads to large scale atmospheric and ocean currents that try to redistribute the energy evenly.
%\end{itemize}
%\end{frame}

%------------------------------------------------------------%
%                                                            %
%                      Reference Slide                       %
%                                                            %
%------------------------------------------------------------%
\begin{frame}[label=x,allowframebreaks]
\frametitle{List of Resources}
\bibliographystyle{amsalpha}
\bibliography{myMasterBib}
\end{frame}

\end{document}

\fig{\includegraphics[width=.9\linewidth, clip=true,trim=0cm 0cm 0cm 0cm]{XXX}}{YYY}{\href{ZZZ}{WWW}}

%------------------------------------------------------------%
%                                                            %
%                           Slide                            %
%                                                            %
%------------------------------------------------------------%
\begin{frame}
\frametitle{XXX}

\begin{columns}
	\column{.5\linewidth}
	LEFT
	\column{.5\linewidth}
	RIGHT
\end{columns}
\end{frame}

\end{document}
